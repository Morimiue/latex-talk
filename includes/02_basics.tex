\section{快速上手}

\subsection{凡例}

\begin{frame}{凡例}
  术语的第一次出现通常会使用\textbf{粗体}。
  网页链接用\ {\tiny \faLink} 符号表示。

  \begin{block}{扩展}
    这里是扩展性的内容,一般不会影响使用,但能了解是最好的。
  \end{block}

  \begin{alertblock}{注意}
    这里是需要注意的内容,与排版中可能出现的问题相关。
  \end{alertblock}

  \begin{exampleblock}{示例}
    这里是示例,通常以代码加效果的形式出现。
  \end{exampleblock}
\end{frame}

\subsection{安装指南 \& 不安装指南}

\begin{frame}[fragile]{安装指南}
  \begin{enumerate}
    \item 下载:推荐下载\TeX{} Live 的 ISO 镜像离线安装
          \begin{itemize}
            \item 清华大学开源软件镜像站 \link{https://mirrors.tuna.tsinghua.edu.cn/CTAN/systems/texlive/Images/texlive.iso}
            \item 中国科学技术大学开源软件镜像站 \link{https://iso.mirrors.ustc.edu.cn/CTAN/systems/texlive/Images/texlive.iso}
            \item 或者其他镜像站
          \end{itemize}
    \item 安装:建议新手安装完整版
          \begin{itemize}
            \item 以管理员权限运行 |install-tl-windows.bat|
            \item 一路点击“下一步”
            \item 保持耐心
          \end{itemize}
    \item 编辑器:再安装一个合适的编辑器
          \begin{itemize}
            \item 专用型:TeXworks、TeXstudio、TeXShop、Kile……
            \item 通用型:Visual Studio Code、Vim、Emacs……
          \end{itemize}
  \end{enumerate}
  更详细的安装教程:\link{https://mirrors.tuna.tsinghua.edu.cn/CTAN/info/install-latex-guide-zh-cn/install-latex-guide-zh-cn.pdf}

  \note{
    \begin{itemize}
      \item 有很多人推荐其他的 \LaTeX{} 发行版,TeX Live 的确有种种缺点:
            \begin{itemize}
              \item 默认占用空间太大,会安装一堆你可能永远用不到的内容
              \item 对宏包的管理不够方便
              \item Windows 下 TeX Live 版本升级不方便
              \item ……
            \end{itemize}
      \item 但我为新手只推荐 TeX Live 完整版,主要是因为:
            \begin{itemize}
              \item 包含所有常用的宏包、文档、字体等,避免后续的麻烦
              \item 用户基数大,这意味着当你遇到问题时,可以更快地找到解决方案
            \end{itemize}
    \end{itemize}
  }
\end{frame}

\begin{frame}{不安装指南}
  \begin{itemize}
    \item 懒得安装?没关系!
    \item 已经 2202 年了,云端服务可能更好用
    \item 免去安装、配置、升级等一系列烦恼,可以多人协作
    \item 国际平台:Overleaf \link{https://cn.overleaf.com/}
          \begin{itemize}
            \item 老牌服务,成熟稳定,模板丰富
            \item 缺少部分中文字体(例如 Windows 自带的中易系列字体)
          \end{itemize}
    \item 国内平台:TeXPage \link{https://www.texpage.com}
          \begin{itemize}
            \item 默认支持更多的中文字体
            \item 自带免费的数学符号选择器
            \item 不够成熟,收录的模板也较少
          \end{itemize}
  \end{itemize}

  \note{
    \begin{itemize}
      \item 推荐先使用在线环境体验 \LaTeX{}
    \end{itemize}
  }
\end{frame}

\subsection{基本使用方法}

\begin{frame}[fragile]{简单的英文文档}
  \begin{exampleblock}{示例}
    \begin{texcode}[gobble=6,basicstyle=\small\ttfamily,emph={[1]document}, emph={[2]article}]
      % 用 pdfLaTeX、XeLaTeX 或 LuaLaTeX 编译
      \documentclass{article}

      \begin{document}
        Hello World!
      \end{document}
    \end{texcode}
  \end{exampleblock}
\end{frame}

\begin{frame}[fragile]{\LaTeX{} 文档的组成部分}
  \begin{enumerate}
    \item 导言区
          \begin{itemize}
            \item 以 |\documentclass{文档类名}| 开头
            \item \textbf{文档类}定义了这个文档的基本格式
            \item 在导言区,你还可以:
                  \begin{itemize}
                    \item 加载宏包,这样就能在正文中使用更多功能
                    \item 调整格式,因为文档类不一定满足你的需求
                  \end{itemize}
            \item \alert{导言区绝不能出现正文,否则会报错}
          \end{itemize}
    \item 正文区
          \begin{itemize}
            \item 以 |\begin{document}| 开头
                    \item 以 |\end{document}| 结尾
            \item 所有正文内容都要放在这里
            \item 原则上来说,正文区只负责内容,不要在此调整格式
          \end{itemize}
  \end{enumerate}
\end{frame}

\begin{frame}[fragile]{简单的中文文档}
  \begin{exampleblock}{示例}
    \begin{texcode}[gobble=6,basicstyle=\small\ttfamily,emph={[1]document}, emph={[2]ctexart}]
      % 用 XeLaTeX 或 LuaLaTeX 编译
      \documentclass{ctexart}

      \begin{document}
        你好,世界!
      \end{document}
    \end{texcode}
  \end{exampleblock}

  \begin{alertblock}{注意}
    \begin{itemize}
      \item 确保文件编码是 UTF-8
      \item 务必使用 |ctexart| 等中文文档类
    \end{itemize}
  \end{alertblock}

  \note{
    \begin{itemize}
      \item 编写中文文档与英文文档的区别:
            \begin{itemize}
              \item 必须使用 \CTeX{} 宏集提供的中文文档类:ctexart、ctexrep、ctexbook 和 ctexbeamer
              \item 加载 xeCJK 等宏包也可以达到效果,但不推荐
              \item 千万不要用一个叫做 \CTeX{} 套装的东西,那玩意儿太古老了
              \item 必须用 \XeLaTeX 等引擎编译
            \end{itemize}
    \end{itemize}
  }
\end{frame}

\begin{frame}[fragile]{基本语法}
  \begin{description}
    \item[注释] 以 |%| 开头,其后所有内容都会被忽略
    \item[命令] 以 |\| 开头,区分大小写,可以提供内容或格式
      \begin{itemize}
        \item |\foo|:有些命令可以不带参数直接使用
        \item |\foo{arg}|:必需参数放在大括号 |{...}| 内
        \item |\foo[bar]{arg}|:可选参数放在中括号 |[...]| 内
      \end{itemize}
    \item[环境] 为一整块内容提供特定的格式
      \begin{texcode}[gobble=8, emph={[1]env}]
        \begin{env}
          ...
        \end{env}
      \end{texcode}
  \end{description}
  \vspace{-1em}
  \begin{alertblock}{注意}
    \begin{itemize}
      \item 有些符号在 \LaTeX{} 中有特殊的作用,因此不能直接输入\\
            需要\textbf{转义}:如用 |\%| 表示 |%|、|\textbackslash| 表示 |\| 等
      \item 连续多个空格 = 单个空格、单个换行符 = 单个空格
            % \item 命令和环境都可以自行定义或修改,但要小心行事
    \end{itemize}
  \end{alertblock}

  \note{
    \begin{itemize}
      \item 一个命令可能既有必需参数,也有可选参数
      \item 有多个参数时,参数的位置是固定的,不能随便放
            % \item 例如,如果一个命令格式是 |\foo[bar]{arg}|,就绝对不能写成 |\foo{arg}[bar]|
    \end{itemize}
  }
\end{frame}

\begin{frame}[fragile]
  \frametitle{谋篇布局}
  \begin{itemize}
    \item 文档部件
          \begin{itemize}
            \item 标题:|\title|、|\author|、|\date| $\rightarrow$ |\maketitle|
            \item 目录:|\tableofcontents|
            \item 章节:|\chapter|、|\section|、|\subsection| 等
            \item 文献:|\bibliography|
          \end{itemize}
    \item 文档划分
          \begin{itemize}
            \item 凤头猪肚豹尾:|\frontmatter|、|\mainmatter|、|\backmatter|
            \item 分文件编译:|\include|、|\input|
          \end{itemize}
  \end{itemize}
\end{frame}

\begin{frame}[fragile]
  \frametitle{文本标记}
  \begin{itemize}
    \item 字体相关
          \begin{itemize}
            \item 字号:|\tiny|、|\small|、|\large|、|\Large| 等
            \item 加粗:|{\bfseries ...}| 或 |\textbf{...}|
            \item 倾斜:|{\itshape ...}| 或 |\textit{...}|
          \end{itemize}
    \item 段落相关
          \begin{itemize}
            \item 换行:|\\|
            \item 换段:空行或 |\par|
            \item 换页:|\newpage|
                  % \item 缩进:|\indent|
            \item 居中:|\centering| 或 |center| 环境
          \end{itemize}
  \end{itemize}
\end{frame}

\begin{frame}[fragile]{一个更完善的文档}
  \vspace{-1em}
  \begin{exampleblock}{示例}
    \begin{texcode}[gobble=6,basicstyle=\small\ttfamily,emph={[1]document}, emph={[2]ctexart}]
      \documentclass{ctexart}

      \title{这是文章标题}
      \author{这是作者}
      \date{这是日期}

      \begin{document}
        % 在导言区定义标题内容后,在正文区生成标题
        \maketitle
        \section{这是第一节的标题}
        你好,世界!

        这是另一段。
      \end{document}

    \end{texcode}
  \end{exampleblock}
\end{frame}

\begin{frame}[fragile]{命令是如何起作用的?}
  \begin{block}{扩展}
    \begin{itemize}
      \item 计算机编程中的\textbf{宏}
            \begin{itemize}
              \item 可以将一段内容替换为另一段内容,这一过程称为\textbf{宏展开}
            \end{itemize}
      \item \TeX{} 就是一种基于宏的系统
            \begin{itemize}
              \item \TeX{} 排版引擎只能解析有限的命令,它们称为\textbf{原始命令}
              \item 其他命令都是宏,最终会被层层展开以供排版引擎处理
              \item \LaTeX{} 提供了额外的宏,各种宏包、用户也可以自定义宏
            \end{itemize}
    \end{itemize}
  \end{block}
  \vspace{-1em}
  \begin{exampleblock}{示例}
    \begin{itemize}
      \item 为什么 |\TeX| 命令能输出错落有致的排版效果(\TeX)?
      \item 因为它会被展开成下面这一堆代码:\\
            \footnotesize |T\kern -.1667em\lower .5ex\hbox {E}\kern -.125emX|
    \end{itemize}
  \end{exampleblock}

  \note{
    \begin{itemize}
      \item 宏可以大大简化代码的编写
    \end{itemize}
  }
\end{frame}

\begin{frame}[fragile]{引擎与格式}
  \vspace{-1em}
  \begin{block}{扩展}
    \begin{itemize}
      \item \textbf{引擎}:\TeX{} 的实现,是幕后真正干活(排版)的程序
            \begin{itemize}
              \item \pdfTeX{}:直接生成 PDF,支持 micro-typography
              \item \XeTeX{}:支持 Unicode、OpenType 与复杂文字编排(CTL)
              \item \LuaTeX{}:支持 Unicode、OpenType,内联 Lua
              \item (u)p\TeX{}:日本方面推动,生成 |.dvi|,(支持 Unicode)
            \end{itemize}
      \item \textbf{格式}:\TeX{} 的语言扩展(命令封装)
            \begin{itemize}
              \item plain \TeX{}:Knuth 同志专用
              \item \LaTeX{}:排版科技类文章的事实标准
              \item Con\TeX{t}:基于 \LuaTeX{} 实现,优雅、易用(吗?)
            \end{itemize}
      \item \textbf{发行版}:把引擎、格式、宏包、文档等各种东西打包到一起
            \begin{itemize}
              \item TeX Live:官方维护,跨平台,首选
              \item MacTeX:约等于 macOS 下的 TeX Live
              \item MiKTeX:个人维护,特点是宏包可以随装随用
                    % \item \textbf{程序}:引擎 + dump 之后的格式代码
                    %       \begin{itemize}
                    %         \item \alert{英文文章:\pdfLaTeX{}、\XeLaTeX{} 或 \LuaLaTeX{}}
                    %         \item \alert{中文文章:\XeLaTeX{} 或 \LuaLaTeX{}}
                    %       \end{itemize}
            \end{itemize}
    \end{itemize}
  \end{block}
\end{frame}

\subsection{常用元素}

\begin{frame}[fragile]{常用元素:列表}
  \begin{exampleblock}{示例}
    \begin{columns}
      \begin{column}{0.45\textwidth}
        \begin{texcode}[gobble=10, emph={[1]enumerate,itemize}]
          \begin{enumerate}
            \item 自由软件的定义
              \begin{itemize}
                \item 自由度〇
                \item 自由度一
                \item 自由度二
                \item 自由度三
              \end{itemize}
            \item 自由和非自由的边界
            \item 自由软件定义的实践
          \end{enumerate}
        \end{texcode}
      \end{column}

      \begin{column}{0.45\textwidth}
        % \footnotesize
        \begin{enumerate}
          \item 自由软件的定义
                \begin{itemize}
                  \item 自由度〇
                  \item 自由度一
                  \item 自由度二
                  \item 自由度三
                \end{itemize}
          \item 自由和非自由的边界
          \item 自由软件定义的实践
        \end{enumerate}
      \end{column}
    \end{columns}
  \end{exampleblock}

  \note{
    \begin{itemize}
      \item enumerate 环境提供有编号列表,itemize 环境提供无编号列表,description 环境提供描述性列表。
      \item 在列表环境中,每个 \backslash item 命令就是一个条目
      \item 各种列表环境之间可以相互嵌套,示例中就演示了 enumerate 和 itemize 环境的嵌套。
    \end{itemize}
  }
\end{frame}

\begin{frame}[fragile]{常用元素:图片(一)}
  \begin{exampleblock}{示例}
    \begin{columns}
      \begin{column}{0.6\textwidth}
        \begin{texcode}[gobble=10, moretexcs={\graphicspath,\includegraphics},emph={[1]figure}, emph={[2]graphicx}]
          \usepackage{graphicx}

          % 必需参数为文件名,后缀可省略
          \includegraphics{figures/logo.png}
        \end{texcode}
      \end{column}

      \begin{column}{0.35\textwidth}
        \centering
        \includegraphics[width=1.0\textwidth]{logo.png}
      \end{column}
    \end{columns}
  \end{exampleblock}

  \begin{alertblock}{注意}
    \begin{itemize}
      \item 这是最简单的插入图片示例,但不能直接用在排版中。
    \end{itemize}
  \end{alertblock}
\end{frame}

\begin{frame}[fragile]{浮动体}
  \begin{itemize}
    \item 上面的用法有什么问题?
          \begin{itemize}
            \item 将图片直接嵌入文本特定位置,可能严重影响排版效果
            \item 没有标题,不符合论文排版要求
            \item 没有编号,无法方便地进行引用
          \end{itemize}
    \item \textbf{浮动体}机制
          \begin{itemize}
            \item 可以自动将图片、表格等大块内容移动到合适的位置
            \item 支持标题和自动编号等功能
            \item \LaTeX{} 的 |figure| 和 |table| 环境都是浮动体
            \item 浮动体环境的可选参数 |[htbp]| 可以进行位置控制
          \end{itemize}
    \item 浮动体的使用技巧
          \begin{itemize}
            \item 不要强求浮动体“乖乖待在插入的位置”
                  \link{https://liam.page/2017/03/11/floats-in-LaTeX-basic}
            \item 避免“下图”“上表”等表述,而要使用“图 1”“表 1”等
            \item 建议写完全文之后统一调整
          \end{itemize}
  \end{itemize}
\end{frame}


\begin{frame}[fragile]{交叉引用}
  \begin{itemize}
    \item \textbf{交叉引用}
          \begin{itemize}
            \item 被引处:使用 |\label| 定义标签
                  \begin{itemize}
                    \item 图表:紧跟在 |\caption| 命令之后
                    \item 章节:紧跟在 |\section| 等章节命令之后
                    \item 公式:任意位置
                  \end{itemize}
            \item 引用处:使用 |\ref|、|\eqref| 等引用标签,自动获取编号
          \end{itemize}
  \end{itemize}
  \vspace{-1em}
  \begin{alertblock}{注意}
    \begin{itemize}
      \item “我这里没有成功显示‘图 1’,而是‘图~?’,这是咋回事?”
            \begin{itemize}
              \item 因为文档引用了自身内容,一次编译无法获取某些信息
              \item 需要多次编译
                    \begin{itemize}
                      \item 对于简单的交叉引用,通常编译两次即可
                      \item 推荐用 \pkg{latexmk} 自动处理,各种在线环境也采用了此工具
                    \end{itemize}
            \end{itemize}
    \end{itemize}
  \end{alertblock}
\end{frame}

\begin{frame}[fragile]{常用元素:图片(二)}
  \begin{exampleblock}{示例}
    \begin{columns}
      \begin{column}{0.6\textwidth}
        \begin{texcode}[gobble=10, moretexcs={\graphicspath,\includegraphics},emph={[1]figure}, emph={[2]graphicx}]
          % 加载图片宏包
          \usepackage{graphicx}
          % 可以统一指定图片路径
          \graphicspath{{./figures/}}

          如图~\ref{fig_logo} 所示。
          \begin{figure}
            \centering
            % 必需参数为文件名,后缀可省略
            % 可选参数可指定尺寸、裁剪等选项
            \includegraphics[...]{logo.png}
            \caption{\LaTeX{} 图标}
            \label{fig_logo}
          \end{figure}
        \end{texcode}
      \end{column}

      \begin{column}{0.35\textwidth}
        如图~\ref{fig_logo_} 所示。
        \begin{figure}
          \centering
          \includegraphics[width=1.0\textwidth]{logo.png}
          \caption{\LaTeX{} 图标}
          \label{fig_logo_}  % Avoid duplicate labels warning
        \end{figure}
      \end{column}
    \end{columns}
  \end{exampleblock}

  \note{
    \begin{itemize}
      \item 这一页幻灯片中涉及的点有很多:
            \begin{itemize}
              \item 加载宏包:宏包能提供更多的宏,方便文档编写,一个宏包通常是为一种特定的功能而设计的
              \item 交叉引用
              \item 浮动体
            \end{itemize}
    \end{itemize}
  }
\end{frame}

\begin{frame}[fragile]{常用元素:图片(三)}
  \begin{itemize}
    \item 外部绘图工具
          \begin{itemize}
            \item Mathematica、MATLAB
            \item PowerPoint、Visio、Adobe Illustrator、Inkscape、Figma 等
            \item Matplotlib、Plotly、Plots.jl、R 等
            \item draw.io \link{https://app.diagrams.net/}、
                  Mathcha \link{https://www.mathcha.io}、
                  ProcessOn \link{https://www.processon.com} 等网站
          \end{itemize}
    \item \TeX{} 内联
          \begin{itemize}
            \item Asymptote
            \item \alert{\pkg{pgf}/\pkg{TikZ}、\pkg{pgfplots}}
          \end{itemize}
    \item 插图格式
          \begin{itemize}
            \item 矢量图:|.pdf|%(不再推荐 |.eps|)
            \item 位图:|.png| 或 |.jpg|
            \item 不(完全)支持 |.svg| 和 |.bmp|
            \item \alert{尽量用矢量图}
          \end{itemize}
    \item 参考:\link{https://www.zhihu.com/question/21664179}
          \link{https://tex.stackexchange.com/q/158668}
          \link{https://tex.stackexchange.com/q/72930}
  \end{itemize}
\end{frame}

\begin{frame}[fragile]{常用元素:表格(一)}
  \begin{exampleblock}{示例}
    \begin{columns}
      \begin{column}{0.5\textwidth}
        \begin{texcode}[gobble=10,emph={[1]tabular}]
          \begin{tabular}{cc}
            标题 & 标题 \\
            内容 & 内容 \\
            ...
            内容 & 内容 \\
          \end{tabular}
        \end{texcode}
      \end{column}

      \begin{column}{0.4\textwidth}
        \scriptsize
        \begin{tabular}{cc}
          命令                 & 渲染结果             \\
          |\&|               & \&               \\
          |\%|               & \%               \\
          |\$|               & \$               \\
          |\#|               & \#               \\
          |\_|               & \_               \\
          |\{|               & \{               \\
          |\}|               & \}               \\
          |\textasciitilde|  & \textasciitilde  \\
          |\textasciicircum| & \textasciicircum \\
          |\textbackslash|   & \textbackslash   \\
        \end{tabular}
      \end{column}
    \end{columns}
  \end{exampleblock}

  \begin{alertblock}{注意}
    \begin{itemize}
      \item 这是最简单的插入表格示例,但不能直接用在排版中。
    \end{itemize}
  \end{alertblock}
\end{frame}

\begin{frame}[fragile]{常用元素:表格(二)}
  \vspace{-1em}
  \begin{exampleblock}{示例}
    \begin{columns}
      \begin{column}{0.5\textwidth}
        \begin{texcode}[gobble=10,moretexcs={\toprule,\midrule,\bottomrule},emph={[1]table,tabular}, emph={[2]booktabs}]
          % 绘制三线表的宏包
          \usepackage{booktabs}

          如表~\ref{tab_escape} 所示。
          \begin{table}
            \caption{\LaTeX{} 转义字符}
            \label{tab_escape}
            \begin{tabular}{cc}
              \toprule
              标题 & 标题 \\
              \midrule
              内容 & 内容 \\
              \bottomrule
            \end{tabular}
          \end{table}
        \end{texcode}
      \end{column}

      \begin{column}{0.4\textwidth}
        \scriptsize
        \LaTeX{} 常用转义字符如表~\ref{tab_escape_} 所示。
        \scriptsize
        \begin{table}
          \caption{\LaTeX{} 转义字符}
          \label{tab_escape_}
          \begin{tabular}{cc}
            \toprule
            命令                 & 渲染结果             \\
            \midrule
            |\&|               & \&               \\
            |\%|               & \%               \\
            |\$|               & \$               \\
            |\#|               & \#               \\
            |\_|               & \_               \\
            |\{|               & \{               \\
            |\}|               & \}               \\
            |\textasciitilde|  & \textasciitilde  \\
            |\textasciicircum| & \textasciicircum \\
            |\textbackslash|   & \textbackslash   \\
            \bottomrule
          \end{tabular}
        \end{table}
      \end{column}
    \end{columns}
  \end{exampleblock}

  \note{
    \begin{itemize}
      \item booktabs 宏包定义了不同横线的命令,用于绘制三线表
      \item 类似插入图片,注意本示例中对浮动体和交叉引用的使用
    \end{itemize}
  }
\end{frame}

\begin{frame}[fragile]{常用元素:表格(三)}
  \begin{itemize}
    \item |tabular| 环境配合相关宏包可以实现更多表格功能
    \item 手动绘制表格确实比较令人头疼,且较难维护
    \item 可以使用在线工具生成表格代码:Tables Generator \link{https://www.tablesgenerator.com/}
  \end{itemize}
  \begin{block}{扩展}
    \begin{itemize}
      \item 除了基本的 |tabular| 环境外,也有宏包提供其他表格环境
      \item 特别推荐 \pkg{tabularray} 宏包
            \begin{itemize}
              \item 兼容性好
              \item 功能强大,各种需求皆可轻松实现
            \end{itemize}
    \end{itemize}
  \end{block}
\end{frame}
