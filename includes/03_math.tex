\section{公式}

\subsection{数学模式}

\begin{frame}[fragile]{数学模式}
  \begin{itemize}
    \item 一切公式都要在\textbf{数学模式}下输入
          \begin{itemize}
            \item 提供了公式排版所需的命令和特性
            \item 数学模式的字体设置独立于文本模式
            \item 间距根据符号类型自动调整,空格会被忽略,空行会报错
          \end{itemize}
    \item \textbf{行内公式}
          \begin{itemize}
            \item 公式与文字混排
            \item 用一对美元符号包裹:|$...$|
            \item 示例:理想气体状态方程可以写为 $PV=nRT$, 其中 $P$、$V$ 和 $T$ 分别是压强、体积和绝对温度
          \end{itemize}
    \item \textbf{行间公式}
          \begin{itemize}
            \item 公式单独成行
            \item 无编号:|\[...\]| 或 |equation*| 环境
            \item 带编号:|equation| 环境
          \end{itemize}
  \end{itemize}
\end{frame}

\subsection{公式排版}

\begin{frame}[fragile]{结构}
  \begin{itemize}
    \item 上标:|^{上标}|
    \item 下标:|_{下标}|
    \item 分式:|\frac{分子}{分母}|
    \item 根式:|\sqrt[根指数]{根底数}|
  \end{itemize}

  \begin{exampleblock}{示例}
    \begin{columns}
      \begin{column}{0.7\textwidth}
        \begin{texcode}[gobble=10, emph={[1]enumerate,itemize}]
          % gather 是一个常用的多行公式环境
          \begin{gather*}
            (\sqrt[n]{a})^{n}=a \\
            \log_{a}b=\frac{\log_{c}b}{\log_{c}a}
          \end{gather*}
        \end{texcode}
      \end{column}

      \begin{column}{0.25\textwidth}
        \vspace*{-2em}
        \begin{gather*}
          (\sqrt[n]{a})^{n}=a \\
          \log_{a}b=\frac{\log_{c}b}{\log_{c}a}
        \end{gather*}
      \end{column}
    \end{columns}
  \end{exampleblock}
\end{frame}


\begin{frame}[fragile]{括号(定界符)}
  \begin{itemize}
    \item 基本括号
          \begin{itemize}
            \item |(...)|、|[...]|、|\{...\}|
            \item 绝对值、范数:|...| 或 |\vert...\vert|、|\Vert...\Vert|
            \item 尖括号:|<...>| 或 |\langle...\rangle|
          \end{itemize}
    \item 自动调节大小
          \begin{itemize}
            \item |\left(...\right)| 等
            \item 自动匹配内部公式的尺寸
          \end{itemize}
    \item 手动调节大小
          \begin{itemize}
            \item 只有 4 + 1 档:|\big|、|\Big|、|\bigg|、|\Bigg|
            \item 声明左中右:|\bigl|、|\bigm|、|\bigr| 等
            \item 示例:$\Biggl( \biggl( \Bigl( \bigl( () \bigr) \Bigr) \biggr) \Biggr)$
          \end{itemize}
  \end{itemize}

  \note{
    \begin{itemize}
      \item 尽管手动调节时只有几个尺寸可选,但自动调节时大部分定界符都可以无限延伸,实现原理是用多个“零件”拼接起来
    \end{itemize}
  }
\end{frame}

\begin{frame}[fragile]{数学符号}
  \begin{itemize}
    \item 寻找符号
          \begin{itemize}
            \item 最常用的额外字体包:\pkg{amssymb}
            \item 常用符号表:各种入门教程和编辑器通常都会给出
            \item 更多符号表:\textit{The Comprehensive \LaTeX{} Symbol List}
                  \link{https://ctan.org/pkg/comprehensive}
            \item 手写识别(有趣但不全):Detexify \link{http://detexify.kirelabs.org}
          \end{itemize}
    \item 数学字体
          \begin{itemize}
            \item Times New Roman 字体:\pkg{newtxmath} 宏包
                  % \item \alert{不要用 \pkg{times} 和 \pkg{mathptmx} 宏包}
                  % \item 加粗:使用 \pkg{bm} 宏包的 |\bm| 命令(|\mathbf| 只有直立的字母)
          \end{itemize}
  \end{itemize}
  \vspace{-1em}
  \begin{block}{扩展}
    \begin{itemize}
      \item 新方案:\pkg{unicode-math}
            \link{https://stone-zeng.github.io/2020-05-02-use-opentype-fonts-iii}
            \begin{itemize}
              \item 符号、字体、样式精调的一揽子解决方案
              \item 彻底修改底层,不可与传统方案混用
            \end{itemize}
    \end{itemize}
  \end{block}
\end{frame}

% \begin{frame}[fragile]{结构}
%   \begin{itemize}
%     \item 上下标
%           \begin{itemize}
%             \item |^| 和 |_|:|f^ab| 和 |f^{ab}|,|e^x^2|、|{e^x}^2| 和 |e^{x^2}|
%           \end{itemize}
%     \item 分式
%           \begin{itemize}
%             \item |\frac{<分子>}{<分母>}|
%           \end{itemize}
%     \item 根式
%           \begin{itemize}
%             \item |\sqrt[<次数>]{<内容>}|
%             \item 复杂情况改用分数指数:|{...}^{1/n}|
%           \end{itemize}
%     \item 矩阵与行列式
%           \begin{itemize}
%             \item |matrix|、|pmatrix|、|vmatrix| 等环境
%             \item 语法类似表格:|&| 分列,|\\| 换行
%             \item 推荐 \pkg{physics} 宏包
%           \end{itemize}
%   \end{itemize}
% \end{frame}

\begin{frame}[fragile]{公式排版技巧}
  \begin{itemize}
    \item 建议始终调用 \pkg{amsmath} 宏包
    \item \lstinline[style=style@inline]|\(no)limits| 命令可以调整积分、求和、极限等元素(|\int|、|\sum|、|\lim|)的显示方式
          \begin{itemize}
            \item $\sum\limits_{i=0}^n {x_i}$ 对比 $\sum\nolimits_{i=0}^n {x_i}$
          \end{itemize}
    \item 小分式、行内分式不好看:改用 |a/b|、或改用行间公式
    \item 手动调节括号大小往往比自动调节更好看
    \item \alert{不建议用 MathType 生成 \LaTeX{} 公式}
    \item \alert{不要用 \texttt{\$\$...\$\$}}
          % \item \alert{不推荐 \texttt{\textbackslash dfrac}}
  \end{itemize}
\end{frame}
